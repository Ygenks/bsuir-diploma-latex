\newdimen\origiwstr
\origiwstr=\fontdimen3\font
\fontdimen3\font=2\origiwstr
\section{ТЕХНИКО-ЭКОНОМИЧЕСКОЕ ОБОСНОВАНИЕ РАЗРАБОТКИ АППАРАТНОЙ СИСТЕМЫ ОБРАБОТКИ ВИДЕОПОТОКА}
\fontdimen3\font=\origiwstr
\label{sec:economics}


% \FPeval{\totalProgramSize}{15680}
% \FPeval{\totalProgramSizeCorrected}{8650}

% \FPeval{\normativeManDays}{224}

% \FPeval{\additionalComplexity}{0.12}
% \FPeval{\complexityFactor}{clip(1 + \additionalComplexity)}

% \FPeval{\stdModuleUsageFactor}{0.7}
% \FPeval{\originalityFactor}{0.7}

% \FPeval{\adjustedManDaysExact}{clip( \normativeManDays * \complexityFactor * \stdModuleUsageFactor * \originalityFactor )}
% \FPround{\adjustedManDays}{\adjustedManDaysExact}{0}

% \FPeval{\daysInYear}{365}
% \FPeval{\redLettersDaysInYear}{9}
% \FPeval{\weekendDaysInYear}{104}
% \FPeval{\vocationDaysInYear}{21}
% \FPeval{\workingDaysInYear}{ clip( \daysInYear - \redLettersDaysInYear - \weekendDaysInYear - \vocationDaysInYear ) }

% \FPeval{\developmentTimeMonths}{3}
% \FPeval{\developmentTimeYearsExact}{clip(\developmentTimeMonths / 12)}
% \FPround{\developmentTimeYears}{\developmentTimeYearsExact}{2}
% \FPeval{\requiredNumberOfProgrammersExact}{ clip( \adjustedManDays / (\developmentTimeYears * \workingDaysInYear) + 0.5 ) }

% % тут должно получаться 2 ))
% \FPtrunc{\requiredNumberOfProgrammers}{\requiredNumberOfProgrammersExact}{0}

% \FPeval{\tariffRateFirst}{600000}
% \FPeval{\tariffFactorFst}{3.04}
% \FPeval{\tariffFactorSnd}{3.48}


% \FPeval{\employmentFstExact}{clip( \adjustedManDays / \requiredNumberOfProgrammers )}
% \FPtrunc{\employmentFst}{\employmentFstExact}{0}

% \FPeval{\employmentSnd}{clip(\adjustedManDays - \employmentFst)}


% \FPeval{\workingHoursInMonth}{160}
% \FPeval{\salaryPerHourFstExact}{clip( \tariffRateFirst * \tariffFactorFst / \workingHoursInMonth )}
% \FPeval{\salaryPerHourSndExact}{clip( \tariffRateFirst * \tariffFactorSnd / \workingHoursInMonth )}
% \FPround{\salaryPerHourFst}{\salaryPerHourFstExact}{0}
% \FPround{\salaryPerHourSnd}{\salaryPerHourSndExact}{0}

% \FPeval{\bonusRate}{1.5}
% \FPeval{\workingHoursInDay}{8}
% \FPeval{\totalSalaryExact}{clip( \workingHoursInDay * \bonusRate * ( \salaryPerHourFst * \employmentFst + \salaryPerHourSnd * \employmentSnd ) )}
% \FPround{\totalSalary}{\totalSalaryExact}{0}

% \FPeval{\additionalSalaryNormative}{20}

% \FPeval{\additionalSalaryExact}{clip( \totalSalary * \additionalSalaryNormative / 100 )}
% \FPround{\additionalSalary}{\additionalSalaryExact}{0}

% \FPeval{\socialNeedsNormative}{0.5}
% \FPeval{\socialProtectionNormative}{34}
% \FPeval{\socialProtectionFund}{ clip(\socialNeedsNormative + \socialProtectionNormative) }

% \FPeval{\socialProtectionCostExact}{clip( (\totalSalary + \additionalSalary) * \socialProtectionFund / 100 )}
% \FPround{\socialProtectionCost}{\socialProtectionCostExact}{0}

% \FPeval{\taxWorkProtNormative}{4}
% \FPeval{\taxWorkProtCostExact}{clip( (\totalSalary + \additionalSalary) * \taxWorkProtNormative / 100 )}
% \FPround{\taxWorkProtCost}{\taxWorkProtCostExact}{0}
% \FPeval{\taxWorkProtCost}{0} % это считать не нужно, зануляем чтобы не менять формулы

% \FPeval{\stuffNormative}{3}
% \FPeval{\stuffCostExact}{clip( \totalSalary * \stuffNormative / 100 )}
% \FPeval{\stuffCost}{\stuffCostExact}

% \FPeval{\timeToDebugCodeNormative}{15}
% \FPeval{\reducingTimeToDebugFactor}{0.3}
% \FPeval{\adjustedTimeToDebugCodeNormative}{ clip( \timeToDebugCodeNormative * \reducingTimeToDebugFactor ) }

% \FPeval{\oneHourMachineTimeCost}{5000}

% \FPeval{\machineTimeCostExact}{ clip( \oneHourMachineTimeCost * \totalProgramSizeCorrected / 100 * \adjustedTimeToDebugCodeNormative ) }
% \FPround{\machineTimeCost}{\machineTimeCostExact}{0}

% \FPeval{\businessTripNormative}{15}
% \FPeval{\businessTripCostExact}{ clip( \totalSalary * \businessTripNormative / 100 ) }
% \FPround{\businessTripCost}{\businessTripCostExact}{0}

% \FPeval{\otherCostNormative}{20}
% \FPeval{\otherCostExact}{clip( \totalSalary * \otherCostNormative / 100 )}
% \FPround{\otherCost}{\otherCostExact}{0}

% \FPeval{\overheadCostNormative}{100}
% \FPeval{\overallCostExact}{clip( \totalSalary * \overheadCostNormative / 100 )}
% \FPround{\overheadCost}{\overallCostExact}{0}

% \FPeval{\overallCost}{clip( \totalSalary + \additionalSalary + \socialProtectionCost + \taxWorkProtCost + \stuffCost + \machineTimeCost + \businessTripCost + \otherCost + \overheadCost ) }

% \FPeval{\supportNormative}{30}
% \FPeval{\softwareSupportCostExact}{clip( \overallCost * \supportNormative / 100 )}
% \FPround{\softwareSupportCost}{\softwareSupportCostExact}{0}


% \FPeval{\baseCost}{ clip( \overallCost + \softwareSupportCost ) }

% \FPeval{\profitability}{35}
% \FPeval{\incomeExact}{clip( \baseCost / 100 * \profitability )}
% \FPround{\income}{\incomeExact}{0}

% \FPeval{\estimatedPrice}{clip( \income + \baseCost )}

% \FPeval{\localRepubTaxNormative}{3.9}
% \FPeval{\localRepubTaxExact}{clip( \estimatedPrice * \localRepubTaxNormative / (100 - \localRepubTaxNormative) )}
% \FPround{\localRepubTax}{\localRepubTaxExact}{0}
% \FPeval{\localRepubTax}{0}

% \FPeval{\ndsNormative}{20}
% \FPeval{\ndsExact}{clip( (\estimatedPrice + \localRepubTax) / 100 * \ndsNormative )}
% \FPround{\nds}{\ndsExact}{0}


% \FPeval{\sellingPrice}{clip( \estimatedPrice + \localRepubTax + \nds )}

% \FPeval{\taxForIncome}{18}
% \FPeval{\incomeWithTaxes}{clip(\income * (1 - \taxForIncome / 100))}
% \FPround\incomeWithTaxes{\incomeWithTaxes}{0}

%%%%%%%%%%%%%%%%%%%%%%%%%%%%%%%%%%%%%%%%%%%%%%%%%%

\subsection{Характеристика аппаратной системы обработки видеопотока}
\label{sec:economics:characteristics}

Обработка видеопотока --- сложная в техническом плане задача. Её решение сопровождается
применением широкого спектра инженерных средств. Изменчивость параметров видео во
времени не позволяют конструировать узкоспециализированные устройства или программы по его
обработке, поэтому часто применяется системный подход.

Достоинствами рассматриваемой системы являются модульность и гибкость, возможность
смены алгоритмов обработки изображений во время работы системы, поддержка работы в приложениях,
требующих предсказуемого времени работы его компонентов, а также обладает возможностью
обрабатывать видео выского разрешения (HDTV и выше).

Применение данной системы позвляет рационализировать затраты на оборудования для обработки видео,
позволяя потребителю дополнять существующий конвейер своими блоками обработки, не приобретая
для каждой новой операции отдельное устройство. В случае работы системы в встраиваемом виде
минизирует энергопотребление конечного продукта и его размеры, сохраняя при этом высокую скорость работы.

Система рассчитана на применение в различных отраслях, но особенно хорошо проявляет себя в машиностроении,
авиастроении, космической промышленности и разработке \en{embedded} устройств.

\subsection{Расчёт стоимостной оценки затрат}
\label{sec:economics:cost_estimation}

\subsubsection{Расчёт затрат на разработку аппаратной части}
\label{sec:economics:cost_estimation:hardware}

Расчёт затрат на зарботную плату разработчиков аппаратной части представлен в таблице~\ref{table:economics:cost_estimation:hardware:employee}

\begin{table}[ht]
  \caption{Расчёт основной зарботной платы исполнителей}
  \label{table:economics:cost_estimation:hardware:employee}
  \begin{tabular}{| >{\centering}m{0.4\textwidth}
                  | >{\centering}m{0.18\textwidth}
                  | >{\centering}m{0.15\textwidth}
                  | >{\centering\arraybackslash}m{0.15\textwidth}|}
   \hline
    Категория исполнителя & Эффективный фонд времени работы, дн. & Дневная тарифная ставка, руб. & Тарифная заработная плата, руб. \\
   \hline
    Инженер-проекта & $ 20 $ & $ 80 $ & $ 1600 $ \\
   \hline
    Инженер-системотехник & $ 20 $ & $ 60 $ & $ 1200 $ \\
   \hline
    Всего & & & $ 2800 $ \\
   \hline
    Премия (50 \%) & & & $ 1400 $ \\
   \hline
    Основная заработная плата & & & $ 4200 $  \\
   \hline
  \end{tabular}
\end{table}

Расчёт затрат на разработку аппаратной части представлен в таблице~\ref{table:economics:cost_estimation:hardware:employee_total}

\begin{table}[ht]
  \caption{Расчёт затрат на разработку аппаратной части}
  \label{table:economics:cost_estimation:hardware:employee_total}
  \begin{tabular}{| >{\centering}m{0.3\textwidth}
                  | >{\centering}m{0.4\textwidth}
                  | >{\centering\arraybackslash}m{0.2\textwidth}|}
   \hline
    Наименование статьи затрат & Расчёт & Значение, руб. \\
   \hline
    Основная заработная плата разработчиков & См. таблицу~\ref{table:economics:cost_estimation:hardware:employee} & $ 4200 $ \\
   \hline
    Дополнительная зарплата & $ \dfrac{4200 \cdot 20}{100} $ & $ 840 $ \\
   \hline
    Отчисления на социальные нужды & $ \dfrac{(4200 + 840) \cdot 34,6}{100} $ & $ 1743 $ \\
   \hline
  \end{tabular}
\end{table}

Расчёт затрат на создание прототипа системы обработки видеопотока представлен в таблице~\ref{table:economics:cost_estimation:hardware:prerequisites}

\begin{table}[ht]
  \caption{Расчёт затрат на комплектующие и средства разработки}
  \label{table:economics:cost_estimation:hardware:prerequisites}
  \begin{tabular}{| >{\centering}m{0.3\textwidth}
                  | >{\centering}m{0.15\textwidth}
                  | >{\centering}m{0.15\textwidth}
                  | >{\centering}m{0.15\textwidth}
                  | >{\centering\arraybackslash}m{0.15\textwidth}|}
  \hline
    Наименование оборудования или ПО & Тип, марка & Количество, шт. & Цена за единицу, руб. & Стоимость, руб. \\
  \hline
    Xilinx Artix-7 FPGA Evaluation Kit & AC701 & $ 1 $ & $ 2619 $ & $ 2619 $ \\
  \hline
    Модуль камеры & MT9T031 & $ 1 $ & $ 43 $ & $ 43 $ \\
  \hline
    САПР Vivado Design Suite & HL Design Edition & $ 1 $ & $ 2023 $ & $ 2023 $ \\
  \hline
    Всего & &  &  & $ 4685 $ \\
  \hline
    Транспортно-заготовительные расходы для комплектующих (20 \%)  & & & & $ 937 $ \\
  \hline
    Всего с транспортно-заготовительными расходами & & & & $ 5622 $ \\
  \hline
  \end{tabular}
\end{table}

Данные затраты являются единовременными затратами на проектирование, в дальнейшем предприятие может использовать
спроектированную систему на базе любой микросхемы FPGA седьмой серии компании Xilinx. В данном случае
необходимо спроектировать минимальную схему, содержащую следующую периферию:

\begin{itemize}
  \item модуль КМОП или ПЗС матрицы с микроконтроллером;
  \item устройство вывода HDMI;
  \item I2C контроллер;
  \item DDR-память для хранения кадров поступающих из камеры;
\end{itemize}

\subsubsection{Расчёт затрат на разработку программной части}
\label{sec:economics:cost_estimation:software}

Разработка программной части включает в себя написание конфигурационного кода
для контроллеров HDMI, I2C и модуля камеры, проектирование API для
передачи кадров в оперативную память и последующую их передачу на HDMI-порт.

Расчёт затрат на зарботную плату разработчиков программной части представлен в таблице~\ref{table:economics:cost_estimation:software:employee}

\begin{table}[ht]
  \caption{Расчёт основной зарботной платы исполнителей}
  \label{table:economics:cost_estimation:software:employee}
  \begin{tabular}{| >{\centering}m{0.4\textwidth}
                  | >{\centering}m{0.18\textwidth}
                  | >{\centering}m{0.15\textwidth}
                  | >{\centering\arraybackslash}m{0.15\textwidth}|}
   \hline
    Категория исполнителя & Эффективный фонд времени работы, дн. & Дневная тарифная ставка, руб. & Тарифная заработная плата, руб. \\
   \hline
    Ведущий инженер-программист & $ 15 $ & $ 80 $ & $ 1200 $ \\
   \hline
    Инженер-программист 2.к & $ 15 $ & $ 60 $ & $ 900 $ \\
   \hline
    Всего & & & $ 2100 $ \\
   \hline
    Премия (50 \%) & & & $ 1050 $ \\
   \hline
    Основная заработная плата & & & $ 3150 $  \\
   \hline
  \end{tabular}
\end{table}

Расчёт затрат на разработку программной части представлен в таблице~\ref{table:economics:cost_estimation:software:employee_total}

\begin{table}[ht]
  \caption{Расчёт затрат на разработку программной части}
  \label{table:economics:cost_estimation:software:employee_total}
  \begin{tabular}{| >{\centering}m{0.3\textwidth}
                  | >{\centering}m{0.4\textwidth}
                  | >{\centering\arraybackslash}m{0.2\textwidth}|}
   \hline
    Наименование статьи затрат & Расчёт & Значение, руб. \\
   \hline
    Основная заработная плата разработчиков & См. таблицу~\ref{table:economics:cost_estimation:software:employee} & $ 3150 $ \\
   \hline
    Дополнительная зарплата & $ \dfrac{3150 \cdot 20}{100} $ & $ 630 $ \\
   \hline
    Отчисления на социальные нужды & $ \dfrac{(3150 + 630) \cdot 34,6}{100} $ & $ 1307 $ \\
   \hline
  \end{tabular}
\end{table}

\subsubsection{Расчёт капитальных вложений}
\label{sec:economics:capital_investment}

Капитальные вложения рассчитываются с учётом получения прибыли с продажи готовой системы,
а не минимально возможной платы, т.к. последнее потребует организации сложного производства.
Расчёт капитальных вложений проведён в соответствии с таблицей~\ref{table:economics:capital_investment_total}

\begin{table}[ht!]
  \caption{Капитальные вложения на разработку системы}
  \label{table:economics:capital_investment_total}
  \begin{tabular}{| >{\centering}m{0.4\textwidth}
                  | >{\centering}m{0.3\textwidth}
                  | >{\centering\arraybackslash}m{0.2\textwidth}|}
   \hline
    Наименование затрат & Расчёт & Сумма, руб. \\
   \hline
    Затраты на разработку аппаратной части & Таблица~\ref{table:economics:cost_estimation:hardware:employee_total} & $ 6783 $ \\
   \hline
    Затраты на разработку программной части & Таблица~\ref{table:economics:cost_estimation:software:employee_total}  & $ 5087 $ \\
   \hline
    Затраты на комплектующие и средства разработки & Таблица~\ref{table:economics:cost_estimation:hardware:prerequisites}  & $ 5622 $ \\
   \hline
    Всего &  & $ 17492 $ \\
   \hline
    Накладные расходы (50 \%) & $ \dfrac{17492 \cdot 50}{100} $ & $ 8746 $ \\[1cm]
   \hline
    Всего затрат на разработку & $ 17492 + 8746 $ & $ 26238 $ \\
   \hline
    Прибыль (50 \%) & $ \dfrac{26238 \cdot 50}{100} $ & $ 13119 $ \\[1cm]
   \hline
    Отпускная цена & $ 26238 + 13119 $ & $ 39357 $ \\
   \hline
    Налог на добавленную стоимость (20 \%) & $ \dfrac{39357 \cdot 20}{100} $ & $ 7871,4 $ \\[1cm]
   \hline
    Отпускная цена с НДС & $ 39357 + 7871,4 $ & $ 47228,4 $ \\
   \hline
  \end{tabular}
\end{table}


\subsection{Расчёт экономической эффективности разработки системы обработки видеопотока}
\label{sec:economics:efficiency}

Экономическим эффектом у предприятия-разработчика системы является чистая прибыль, остающаяся в распоряжении организации, которая составит:

\begin{equation}
  \label{eq:economics:income}
  \text{П} = 13119 - \dfrac{13119 \cdot 18}{100} = \SI{10757,58}{\text{руб.}}
\end{equation}

Рентабельность затрат на разработку данной системы для организации-разработчика составит:

\begin{equation}
  \label{eq:economics:profit}
  \text{P} = \dfrac{10757,58}{26238} * 100\% = 41\% % какого хрена совпало с методой???
\end{equation}

На основании полученных результатов экономического обоснования можно сделать вывод,
что затраты на разработку и внедрение данной системы являются экономически эффективными
как для предприятия-разработчика, так и для предприятия-заказчика системы. После выполнения
работ предприятие-разработчик получает чистую прибыль в размере 10757,58 руб., при этом
рентабельность разработки и установки составит 41\%.
