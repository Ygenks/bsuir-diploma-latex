\sectioncentered*{Заключение}
\addcontentsline{toc}{section}{ЗАКЛЮЧЕНИЕ}

По результатам выполненеия дипломного проекта была разработана
аппаратная система разработки видеопотока. Данная система предзназначена
для использования в качестве платформы для обработки видео высокого разрешения.

Система позволяет применять произвольные методы обработки видео, работающие как без,
так и с использованием дополнительной памяти. Также поддерживается контроль над состоянием
системы, за счёт применения микропроцессора.

Особенностью, которая отличает данное приложение от аналогов, является модульность,
низкое энергопотребление и высокая скорость работы. Данные качества достигнуты за счёт применения
современной технологической базы в виде FPGA, которая позволяет проектировать произвольную
логику любой сложности, ограниченную лишь внутренними ресурсами микросхемы. Перепрограммируемость
дала возможность производителям микросхем предоставлять переиспользуемые аппаратные блоки,
которые применяются разработчиками с целью существенной экономии времени проектирования.

На основании вышеприведенных сведений, поставленные цели можно считать выполненными в полном объеме.
Дальнейшие планы доработки системы заключаются в введении поддержки полноценного логирования
и перевода программной архитектуры на модель конечного автомата.
