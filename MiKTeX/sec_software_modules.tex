\section{РАЗРАБОТКА ПРОГРАММНЫХ МОДУЛЕЙ}
\label{sec:software_modules}

После проектирования аппаратной базы системы обработки видеопотока
необходимо разработать программный слой проекта. Конфигурация системы
осуществляется при помощи микропроцессора MicroBlaze. Она делится
на два множества: конфигурирование IP-блоков по протоколу AXI4-Lite и
внешней периферии, такой как контроллер HDMI и CMOS-камера, посредством
шины I2C. Рассмотрим подробнее часть функций.


Так как работа с контроллером HDMI происходит по протоколу I2C, то
сперва необходимо задать адрес устройства контроллера на этой шине.
Для этого необходимо проинициализировать блок AXI HDMI TX адресом контроллера,
взаимодействие с блоком происходит по шине AXI4-Lite.
Для этого вызывается библиотечная функция \texttt{HAL\_paltform\_init}, которая
задаёт адрес на шине I2C и адрес устройства сброса.

Для поддержки произвольного адреса контроллера на шине используется макроподстановка,
результат которой зависит от того, объявлен ли произвольный адрес контроллера в файле
BSP (\en{Board Support Package}):
\medskip
\begin{lstlisting}[style=C]
#ifdef XPAR_AXI_IIC_0_BASEADDR
	HAL_platform_init(XPAR_AXI_IIC_0_BASEADDR, XPAR_AXI_PSR_0_BASEADDR);
#else
  	HAL_platform_init(XPAR_AXI_IIC_MAIN_BASEADDR, XPAR_AXI_PSR_0_BASEADDR);
#endif
\end{lstlisting}
\medskip

После инициализации адреса контроллера необходимо рассчитать параметры
частоты для блока AXI CLKGEN, который тактирует контроллер HDMI.
Частота генериуемого тактового сигнала задаётся путём записи нужных
значений в регистры блока управления тактовыми сигналами. Рассмотрим
примитивы по работе с MMCM.

Первым примитивом, необходимым для отслеживания состояния блока, является
чтение из блока управления тактовыми сигналами. Так как работа с блоком происходит
не напрямую, а через аппаратную прослойку, то чтение ргеистров происходит со смещением
от адреса блока CLKGEN:
\medskip
\begin{lstlisting}[style=C]
  static void read(unsigned long register, unsigned long *value)
  {
	*value = Xil_In32(CF_CLKGEN_BASEADDR + register);
  }
\end{lstlisting}
\medskip

Чтение производится по шине AXI4-Lite, а так как процессор MicroBlaze реализует
отображение периферийных устройств в память, то чтение представляет собой не что иное,
как разименование указателя, размером, равным разрядности шине данных:
\medskip
\begin{lstlisting}[style=C]
  static INLINE u32 Xil_In32(UINTPTR Addr)
  {
	return *(volatile u32 *) Addr;
  }
\end{lstlisting}
\medskip

Определение \texttt{CF\_CLKGEN\_BASEADDR}, по аналогии с контроллером I2C, является
стартовым адресом блока CLKGEN на шине AXI4-Lite.

Примитив записи регистров выглядит схожим образом:
\medskip
\begin{lstlisting}[style=C]
  static void write(unsigned long register, unsigned long value)
  {
	Xil_Out32(CF_CLKGEN_BASEADDR + register, value);
  }
\end{lstlisting}
\medskip

Запись значения в память раскрывается схожим образом, при этом
заданный адрес выравнивается кратно размеру шины адреса, после
чего происходит запись:
\medskip
\begin{lstlisting}[style=C]
  static INLINE void Xil_Out32(UINTPTR Addr, u32 Value)
  {
	volatile u32 *LocalAddr = (volatile u32 *)Addr;
	*LocalAddr = Value;
  }
\end{lstlisting}
\medskip

Примитивы чтения и записи не содержат логики обработки ошибок и не учитывают
задержки при применении параметров для блока управления частоты. Поэтому
необходимо обернуть примитивы управляющей логикой. Пример функции чтения
значения регистра блока управления тактовыми сигналами:
\medskip
\begin{lstlisting}[style=C]
  static int mmcm_read(unsigned int reg, unsigned int *val)
  {
	unsigned int reg_val;
	int ret;

	ret = wait_non_busy(axi_clkgen);
	if (ret < 0) {
      return ret;
    }

	reg_val = AXI_CLKGEN_V2_DRP_CNTRL_SEL | AXI_CLKGEN_V2_DRP_CNTRL_READ;
	reg_val |= (reg << 16);

	write(AXI_CLKGEN_V2_REG_DRP_CNTRL, reg_val);

	ret = wait_non_busy(axi_clkgen);
	if (ret < 0) {
      return ret;
    }

	*val = ret;

	return 0;
  }
\end{lstlisting}
\medskip

Чтение значения происходит до истечения программного таймаута, так как
конфигурация выполняется единожды, из-за чего нет необходимости использовать
таймеры. Истечение таймаута может говорить о занятости устройства или аварийном статусе.
При чтении, необходимо считать статус блока управления тактовыми сигналами,
так как тот может быть заблокирован для модификации, например во время смены частоты.
Затем происходит запись управляющих регистров, для выбора канала в MMCM. В данном
случае используется нулевой канал. После чего происходит чтение значения регистра данного
канала.

Запись значения в регистры происходит следующим образом:
\medskip
\begin{lstlisting}[style=C]
  static int mmcm_write(unsigned int reg, unsigned int val, unsigned int mask)
  {
	unsigned int reg_val = 0;
	int ret;

	ret = wait_non_busy(axi_clkgen);
	if (ret < 0) {
      return ret;
    }

	if (mask != 0xffff) {
      mmcm_read(reg, &reg_val);
      reg_val &= ~mask;
	}

	reg_val |= AXI_CLKGEN_V2_DRP_CNTRL_SEL | (reg << 16) | (val & mask);

	write(AXI_CLKGEN_V2_REG_DRP_CNTRL, reg_val);

	return 0;
  }
\end{lstlisting}
\medskip

Функция \texttt{wait\_non\_busy} обеспечивает обнаружение окна доступа к устройству,
возвращая управление, когда блок готов принимать и передавать данные:
\medskip
\begin{lstlisting}[style=C]
  static int wait_non_busy(struct axi_clkgen *axi_clkgen)
  {
	unsigned int timeout = 10000;
	unsigned int val;

	do {
      read(AXI_CLKGEN_V2_REG_DRP_STATUS, &val);
	} while ((val & AXI_CLKGEN_V2_DRP_STATUS_BUSY) && --timeout);

	if (val & AXI_CLKGEN_V2_DRP_STATUS_BUSY)
    return -EIO;

	return val & 0xffff;
  }
\end{lstlisting}
\medskip

Для настройки необходимой частоты требуется задать канал MMCM, на котором
будет сгенерирован сигнал, делитель частоты для данного канала, указать
фильтры, которые будут применены к сгенерированному сигналу, а также
записать, при каких значениях модуль генерации тактовых сигналов
зафиксирует параметры сгенерированной частоты. Функция \texttt{set\_rate}:
\medskip
\begin{lstlisting}[style=C]
  static int set_rate(struct clk_hw *clk_hw, unsigned long rate, unsigned long parent_rate)
  {
    *axi_clkgen = clk_hw_to_axi_clkgen(clk_hw);
	unsigned int d, m, dout;
	unsigned int nocount;
	unsigned int high;
	unsigned int edge;
	unsigned int low;
	uint32_t filter;
	uint32_t lock;

	if (parent_rate == 0 || rate == 0)
    return -EINVAL;

	CLKGEN_calc_params(parent_rate, rate, &d, &m, &dout);

	if (d == 0 || dout == 0 || m == 0)
    return -EINVAL;

	filter = lookup_filter(m - 1);
	lock = lookup_lock(m - 1);

	CLKGEN_calc_clk_params(dout, &low, &high, &edge, &nocount);
	mmcm_write(MMCM_REG_CLKOUT0_1, (high << 6) | low, 0xefff);
	mmcm_write(MMCM_REG_CLKOUT0_2, (edge << 7) | (nocount << 6), 0x03ff);

	CLKGEN_calc_clk_params(d, &low, &high, &edge, &nocount);
	mmcm_write(MMCM_REG_CLK_DIV,
    (edge << 13) | (nocount << 12) | (high << 6) | low, 0x3fff);

	CLKGEN_calc_clk_params(m, &low, &high, &edge, &nocount);
	mmcm_write(MMCM_REG_CLK_FB1,
    (high << 6) | low, 0xefff);
	mmcm_write(MMCM_REG_CLK_FB2,
    (edge << 7) | (nocount << 6), 0x03ff);

	mmcm_write(MMCM_REG_LOCK1, lock & 0x3ff, 0x3ff);
	mmcm_write(MMCM_REG_LOCK2,
    (((lock >> 16) & 0x1f) << 10) | 0x1, 0x7fff);
	mmcm_write(MMCM_REG_LOCK3,
    (((lock >> 24) & 0x1f) << 10) | 0x3e9, 0x7fff);
	mmcm_write(MMCM_REG_FILTER1, filter >> 16, 0x9900);
	mmcm_write(MMCM_REG_FILTER2, filter, 0x9900);

	return 0;
  }
\end{lstlisting}
\medskip

Расчёт параметров производится библиотечными функциями блока генерации тактовых сигналов.
Передаваемые значения коэффициента фильтрации и частоты вычисляется путём обхода таблицы коэффициентов.
Запись предельных значений для частоты производится путём сдвига рассчитанного значения,
с применением маски, т.к. за определённые параметры отвечает не целый регистр, а группа бит,
в которое нужно сместить полученное значение.

Настройка ядра AXI HDMI TX происходит с использованием библиотеки от производителя контроллера,
так как включает примитивы по работе с контроллером HDMI по протоколу I2C.
Для удобства при инициализации сложного устройства используется структура, содержащая
все конфигурируемые параметры. Инициализация блока включает в себя аппаратную и программную часть, и
начинается с установки параметров работы устройства:
\medskip
\begin{lstlisting}[style=C]
  ATV_ERR init_transmitter()
  {
    last_detected_mode = MODE_INVALID;

    trasmitter.changed = TRUE;
    trasmitter.mode = MODE_NONE;
    trasmitter.req_output_mode = OUT_MODE_HDMI;
    trasmitter.bits_per_color = 8;
    trasmitter.format = SDR_444_SEP_SYNC;
    trasmitter.pixel_style = 2;
    trasmitter.alignment = ALIGN_RIGHT;
    trasmitter.pixel_encoding = OUT_ENC_YUV_444;
    trasmitter.in_color_space = TX_CS_RGB;
    trasmitter.out_color_space = TX_CS_RGB;
    trasmitter.audio_interface = TX_SPDIF;
    trasmitter.debug = true;

    mute = MUTE_ENABLE;

    software_init();
    hardware_init();

    return ATVERR_OK;
  }
\end{lstlisting}
\medskip

Данная функция задаёт цветовое пространство входного и выходного потока,
количество бит на канал, кодировку изображения, количество передаваемых
пикселей за такт. После чего выполняется программная и аппаратная инициализация.


Программная инициализация состоит из зануления структуры, которая используется
передатчиком для хранения текущего состояния:
\medskip
\begin{lstlisting}[style=C]
  void software_init()
  {
    memset(&(tx_status), 0, sizeof(TX_STATUS_PKT));

    tx_status.Hpd        = false;
    tx_status.Msen       = false;
  }
\end{lstlisting}
\medskip

Аппаратная инициализация производит запись значений структуры с конфигурационными
параметрами в устройство средствами библиотеки, так же активируя режим
TMDS (\en{Transition-minimized differential signaling}), используемый протоколами
DVI и HDMI для уменьшения помех в сигнале, путём применения дифференциальных сигналов
и избыточного кодирования \en{8b/10b}. Отключается звуковой канал, так как он не
используется в проекте.
\medskip
\begin{lstlisting}[style=C]
  void hardware_init()
  {
    HAL_enable_tx_hpd(true);

    ADIAPI_TxInit(true);

    ADIAPI_TxEnableTmds(true, true);

    ADIAPI_TxSetAvmute(TX_AVMUTE_OFF);
    ADIAPI_TransmitterSetMuteState();

    ADIAPI_TxSetOutputMode(trasmitter.req_output_mode);

    ADIAPI_TxSetInputPixelFormat(transmitter.bits_per_color,
    trasmitter.format,
    trasmitter.pixel_style,
    trasmitter.alignment,
    false,
    false);

    ADIAPI_TxSetOutputPixelFormat(trasmitter.out_pixel_format, true);

    ADIAPI_TxSetCSC(trasmitter.in_color_space, trasmitter.out_color_space);

    ADIAPI_TxSetAudioInterface(trasmitter.audio_interface, AUD_SAMP_PKT, true);

    ADIAPI_TxSetEnabledEvents(TX_EVENT_ALL_EVENTS, false);
    ADIAPI_TxSetEnabledEvents(
    (TX_EVENT)(TX_EVENT_HPD_CHG | TX_EVENT_MSEN_CHG | TX_EVENT_EDID_READY),
    true);

    ADIAPI_TxEnablePackets(PKT_AV_INFO_FRAME, true);
  }
\end{lstlisting}
\medskip

Конфигурация VDMA осщуествляется функцией \texttt{init\_vdma},
которая сбрасывает конечный автомат блока в стартовое состояние. После сброса
осуществляется поиск конфигурационного файла, экспортированного при создании битстрима.
Конфигурация содержит значения включенных параметров блока. В случае успешного поиска
происходит считывание конфигурационного файла. Для установки количества кадров в секунду,
считанных и записанных VDMA используются счётчики кадров. Тело функции \texttt{init\_vdma}:
\medskip
\begin{lstlisting}[style=C]
  int run_triple_frame_buffer(XAxiVdma* InstancePtr, int DeviceId, int hsize,
		int vsize, int buf_base_addr, int number_frame_count,
		int enable_frm_cnt_intr)
        {
          int Status,i;
          XAxiVdma_Config *Config;
          XAxiVdma_FrameCounter FrameCfgPtr;

          /* This is one time initialization of state machine context.
          * In first call it will be done for all VDMA instances in the system.
          */
          if(context_init==0) {
            for(i=0; i < XPAR_XAXIVDMA_NUM_INSTANCES; i++) {
              vdma_context[i].InstancePtr = NULL;
              vdma_context[i].device_id = -1;
              vdma_context[i].hsize = 0;
              vdma_context[i].vsize = 0;
              vdma_context[i].init_done = 0;
              vdma_context[i].buffer_address = 0;
              vdma_context[i].enable_frm_cnt_intr = 0;
              vdma_context[i].number_of_frame_count = 0;

            }
            context_init = 1;
          }

          /* The below initialization will happen for each VDMA. The API argument
          * will be stored in internal data structure
          */

          /* The information of the XAxiVdma_Config comes from hardware build.
          * The user IP should pass this information to the AXI DMA core.
          */
          Config = XAxiVdma_LookupConfig(DeviceId);
          if (!Config) {
            xil_printf("No video DMA found for ID %d\r\n",DeviceId );
            return XST_FAILURE;
          }

          if(vdma_context[DeviceId].init_done ==0) {
            vdma_context[DeviceId].InstancePtr = InstancePtr;

            /* Initialize DMA engine */
            Status = XAxiVdma_CfgInitialize(vdma_context[DeviceId].InstancePtr,
            Config, Config->BaseAddress);
            if (Status != XST_SUCCESS) {
              xil_printf("Configuration Initialization failed %d\r\n",
              Status);
              return XST_FAILURE;
            }

            vdma_context[DeviceId].init_done = 1;
          }

          vdma_context[DeviceId].device_id = DeviceId;
          vdma_context[DeviceId].vsize = vsize;

          vdma_context[DeviceId].buffer_address = buf_base_addr;
          vdma_context[DeviceId].enable_frm_cnt_intr = enable_frm_cnt_intr;
          vdma_context[DeviceId].number_of_frame_count = number_frame_count;
          vdma_context[DeviceId].hsize = hsize * (Config->Mm2SStreamWidth>>3);

          /* Setup the write channel */
          Status = WriteSetup(&vdma_context[DeviceId]);
          if (Status != XST_SUCCESS) {
            xil_printf("Write channel setup failed %d\r\n", Status);
            if(Status == XST_VDMA_MISMATCH_ERROR)
			xil_printf("DMA Mismatch Error\r\n");
            return XST_FAILURE;
          }

          /* The frame counter interrupt is enabled, setting VDMA for same */
          if(vdma_context[DeviceId].enable_frm_cnt_intr) {
            FrameCfgPtr.WriteDelayTimerCount = 1;
            FrameCfgPtr.WriteFrameCount = number_frame_count;

            XAxiVdma_SetFrameCounter(vdma_context[DeviceId].InstancePtr,&FrameCfgPtr);
            /* Enable DMA read and write channel interrupts. The configuration for interrupt
            * controller will be done by application	 */
            XAxiVdma_IntrEnable(vdma_context[DeviceId].InstancePtr,
            XAXIVDMA_IXR_ERROR_MASK |
            XAXIVDMA_IXR_FRMCNT_MASK,XAXIVDMA_WRITE);
            XAxiVdma_IntrEnable(vdma_context[DeviceId].InstancePtr,
            XAXIVDMA_IXR_ERROR_MASK |
            XAXIVDMA_IXR_FRMCNT_MASK,XAXIVDMA_READ);
          } else	{
            /* Enable DMA read and write channel interrupts. The configuration for interrupt
            * controller will be done by application	 */
            XAxiVdma_IntrEnable(vdma_context[DeviceId].InstancePtr,
            XAXIVDMA_IXR_ERROR_MASK,XAXIVDMA_WRITE);
            XAxiVdma_IntrEnable(vdma_context[DeviceId].InstancePtr,
            XAXIVDMA_IXR_ERROR_MASK ,XAXIVDMA_READ);
          }

          return XST_SUCCESS;
        }
\end{lstlisting}
\medskip