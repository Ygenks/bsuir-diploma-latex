\section{СТРУКТУРНОЕ ПРОЕКТИРОВАНИЕ}
\label{sec:structural}

После проработки теоретических вопросов связанных с разработкой системы и
составления списка необходимых требований к системе, необходимо разбить проектируемую систему
на функциональные блоки. Данный подход обеспечивает архитектурную гибкость, позволяя
модифицировать или заменять функциональные блоки без изменения системы в целом.

В разрабатываемой системе обработки видео можно выделить следующие блоки:

\begin{itemize}
  \item блок камеры;
  \item блок Video DMA для входного видеопотока;
  \item блок ОЗУ;
  \item блок Video DMA для выходного видеопотока;
  \item блок микропроцессора;
  \item блок контроллера HDMI;
  \item блок контроллера I2C.
\end{itemize}

Структурная схема, иллюстрирующая перечисленные блоки и связи
между ними приведена на чертеже ГУИР.400201.064 Э1.

Каждый функциональный блок выполняет свой круг задач и взаимодействует с другими блоками
посредством различных аппаратных интерфейсов и протоколов шины.

Рассмотрим более подробно функциональные блоки системы.

\subsection{Блок камеры}
\label{sec:structural:camera}

Блок камеры является инициатором входного видеопотока. Перед началом работы происходит
процесс инициализации, где задаётся разрешение видео, количество кадров в секунду,
различные режимы выборки и т.д. Видеопоток формируется в формате Bayer RGB, в котором
на каждый итоговый пиксель приходится 25\% красных, 25\% синих и 50\% зеленых элементов.
Такая конфигурация светофильтра даёт большую разрешающую способность в зелёной области спектра,
что соответствует особенностям человеческого зрения. Сформированный поток пикселей интерполируется
контроллером камеры, после чего при помощи внешней синхронизации передаётся на преобразователь
RGB в протокол AXI4-Stream(далее AXIS). Модули по обработке видео могут встраиваться в систему как до преобразования в
AXIS, так и на последующих этапах.

% Дописать ещё чё нить

\subsection{Блок Video DMA для входного видеопотока}
\label{sec:structural:vdma_in}
Блок Video DMA(далее VDMA) для входного видеопотока обеспечивает высокоскоростной прямой доступ к памяти
между ОЗУ и видеопериферией по протоколам AXI4 и AXIS. Master-устройством выступает контроллер ОЗУ,
инициируя транзакцию по передаче части кадра из Slave-устройства преобразователя в блок памяти ОЗУ.
Такое решение позволяет исключить микропроцессор из потока обработки данных, где он является
критическим местом по производительности. VDMA поддерживает режим пакетного обмена, дополнительно
ускоряя обмен между камерой и ОЗУ.
