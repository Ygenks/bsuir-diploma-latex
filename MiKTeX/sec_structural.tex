\section{СТРУКТУРНОЕ ПРОЕКТИРОВАНИЕ}
\label{sec:structural}

После проработки теоретических вопросов связанных с разработкой системы и
составления списка необходимых требований к системе, необходимо разбить проектируемую систему
на функциональные блоки. Данный подход обеспечивает архитектурную гибкость, позволяя
модифицировать или заменять функциональные блоки без изменения системы в целом.

В разрабатываемой системе обработки видео можно выделить следующие блоки:

\begin{itemize}
  \item блок камеры;
  \item блок Video In to AXIS;
  \item блок Video DMA для входного видеопотока;
  \item блок ОЗУ;
  \item блок Video DMA для выходного видеопотока;
  \item блок AXI HDMI TX;
  \item блок микропроцессора;
  \item блок контроллера HDMI;
  \item блок контроллера I2C.
\end{itemize}

Структурная схема, иллюстрирующая перечисленные блоки и связи
между ними приведена на чертеже ГУИР.400201.064 Э1.

Каждый функциональный блок выполняет свой круг задач и взаимодействует с другими блоками
посредством различных аппаратных интерфейсов и протоколов шины.

Рассмотрим более подробно функциональные блоки системы.

\subsection{Блок камеры}
\label{sec:structural:camera}

Блок камеры является инициатором входного видеопотока. Перед началом работы происходит
процесс инициализации, где задаётся разрешение видео, количество кадров в секунду,
различные режимы выборки и т.д. Видеопоток формируется в формате Bayer RGB, в котором
на каждый итоговый пиксель приходится 25\% красных, 25\% синих и 50\% зеленых элементов.
Такая конфигурация светофильтра даёт большую разрешающую способность в зелёной области спектра,
что соответствует особенностям человеческого зрения. Сформированный поток пикселей интерполируется
контроллером камеры, после чего при помощи внешней синхронизации передаётся в блок Video In to AXIS.

\subsection{Блок Video In to AXIS}
\label{sec:structural:video_in_to_axis}
Большинство IP-ядер используют формат AXIS для работы с видео, поэтому для последующей обработки
видеопотока камеры необходим преобразователь. Данным преобразователем выступает блок Video In to AXIS,
который преобразовывает видеопоток, состоящий из видеоданных и сигналов синхронизации, в
AXIS-master в соответствии с видео протоколом AXI4-Stream. Блок может принимать
видеоданные от:
\begin{itemize}
  \item DVI;
  \item HDMI;
  \item фотоматриц;
  \item любых других источников параллельного видеосигнала.
\end{itemize}

Если дальнейшая работа с видеосигналом подразумевает наличие тайминговых сигналов синхронизации,
то возможно использование выходных сигналов синхронизации преобразователя при помощи контроллера
синхросигналов.

Такая конструкция позволяет интегрировать блоки обработки видео сразу после его преобразования в AXIS,
что избавляет разработчика от поддержки различных входных видеоформатов, полностью делегируя это преобразователю.

\subsection{Блок Video DMA для входного видеопотока}
\label{sec:structural:vdma_in}
Блок Video DMA (далее VDMA) для входного видеопотока обеспечивает высокоскоростной прямой доступ к памяти
между ОЗУ и видеопериферией по протоколам AXI4 и AXIS. Master-устройством выступает контроллер ОЗУ,
инициируя транзакцию по передаче части кадра из Slave-устройства преобразователя в блок памяти ОЗУ.
Такое решение позволяет исключить микропроцессор из потока обработки данных, где он является
критическим местом по производительности.

VDMA поддерживает динамическую смену частоты для интерфейса AXIS, полезный в тех случаях,
когда источник видеопотока работает нестабильно или переконфигурируется во время работы системы.
Так же VDMA может работать в режиме пакетного обмена, ускоряющий пердачу между источником данных и ОЗУ,
могут использоваться асинхронные синхросигналы для всех интерфейсов семейства AXI4.

\subsection{Блок ОЗУ}
\label{sec:structural:ram}
Размеры кадров видео выского разрешения не позволяют хранить их в статической памяти FPGA. Решением данной проблемы
служит применение динамического ОЗУ. Блок ОЗУ сохраняет данные, поступающие из источника видео (VDMA для входного потока), и отдаёт
их потребителям (VDMA для выходного потока). Объём динамической памяти существенно превышает объём статической ---
1 ГБ против 13 МБ (суммарно для всей микросхемы, включая block RAM, FIFO и т.д.). Благодаря этому в динамической памяти можно разместить
последовательность кадров, что даёт возможность применять более сложные алгоритмы с одновременным доступом к набору кадров.

Управление DRAM осуществляется при помощи контроллера Memory Interface Generator (MIG), который выступает в роли адаптера, скрывая
от разработчика особенности конкретной используемой памяти при помощи \en{Hardware Abstraction Layer (HAL)}. MIG поддерживает современные
стандарты DDR, QDR, RLDRAM и передаёт данные по шине AXI4.

\subsection{Блок Video DMA для выходного видеопотока}
\label{sec:structural:vdma_out}
Описание данного блока во многом схоже с описанием в подразделе \ref{sec:structural:vdma_in},
за исключением инициатора транзакций и направлением передачи данных. Инициатором транзакций выступает
AXI4-Master интерфейс для чтения кадров из памяти в AXIS Master интерфейс, соединённый с блоком AXI HDMI TX.

\subsection{Блок AXI HDMI TX}
\label{sec:structural:axi_hdmi_tx}
Блок AXI HDMI TX ответственнен за подготовку видеоданных к передаче на HDMI контроллер, путём
преобразования видеопотока AXIS в формат HDMI и генерацию синхросигналов развёртки для HDMI контроллера.
Преобразование происходит в несколько этапов:
\begin{enumerate}[label={\arabic*}]
  \item Преобразование цветового пространства из RGB в YCbCr. Может быть опциональным.
  \item Ограничение предельной интенсивности изображения с целью уменьшить шум клиппинга.
    Схожий принцип работы описан в \cite{ycbcr_to_rgb_converter}.
  \item Цветовая субдискретизация для снижения размера цифрового потока видеоданных.
  \item Генерация сигналов синхронизации видеопотока.
  \item Перемежение данных сигналами синхронизации для более компактного пути передачи данных.
  \end{enumerate}

За счёт выделенного генератора синхросигналов частота VDMA и \en{AXI HDMI TX} не обязаны совпадать.
Если видеопоток требует проведения постобработки, то разработчик может поместить блок непосредственно перед
преобразованием, однако такие алгоритмы должны работать непосредственно с потоком данных, без обращения к памяти.

\subsection{Блок микропроцессора}
\label{sec:structural:microprocessor}
Блок микропроцессора выступает главным управляющим блоком в системе.

Так как серия Artix не содержит микропроцессорных ядер, в отличие от линейки ZYNQ,
остаётся два варианта реализации микропроцессора --- внешний к плате микроконтроллер, подключаемый
посредством выводов платы, либо soft-core микропроцессор. В первом подходе отсутствует гибкость и создаются
дополнительные сложности с проработкой интерфейсов между двумя физически независимыми платами, поэтому
рассматривается только второй подход.

Самым современным soft-core процессором от компании Xilinx является MicroBlaze. Данный процессор реализуется
при помощи КЛБ и блоков памяти микросхемы FPGA. С точки зрения архитектуры, MicroBlaze похож на RISC DLX,
выполняя по одной инструкции за такт, за исключением отдельных случаев.

MicroBlaze имеет универсальные средства связи с периферией. Доступ к внутренней памяти осуществляется по специальной
шине LMB, снижая нагрузку на другие шины. Доступны различные параметры конфигурирования MicroBlaze:
размер кэш памяти, длина конвейера (3 или 5 уровней), встроенная периферия, блок управления памятью (MMU),
шинные интерфейсы и так далее. Без MMU на MicroBlaze может работать операционная система с упрощенными системами
защиты и виртуальной памяти (μClinux, FreeRTOS). С поддержкой MMU возможна работа операционных систем, которым необходима
аппаратная поддержка страничной организации памяти и защиты (ядро Linux), хотя производительность таких систем существенно ниже,
чем у упомянутого ранее ZYNQ со встроенным ARM процессором.

Связь с большинством блоков осуществляется посредством шины AXI4-Lite. Данная шина является упрощенным
вариантом шины AXI4, которая используется для передачи данных без жёстких требований по скорости, коррекции ошибок,
большому количеству режимов передачи. При помощи AXI4-Lite управляющая программа микропроцессора конфигурирует регистры
IP-блоков, тем самым инициализируя систему.
% Если какие-то блоки не могут быть проинициализированы посредством AXI4-Lite,
% то микропроцессор может соединяться с I2C либо SPI контроллерами, или с произвольной шиной с реализацией модуля на RTL.

% К обработке видео последовательными алгоритмами может привлекаться микропроцессор, подключая необходимое количество DSP блоков
% и более точно настраивая параметры производительности, вплоть до установки ОС, вместо bare-metal приложения.

\subsection{Блок контроллера HDMI}
\label{sec:structural:hdmi_controller}
Передача обработанного видеосигнала на HDMI интерфейс осуществляется контроллером HDMI.
В качестве контроллера используется ADV7511 компании Analog Devices, рассмотренный в разделе
\ref{sub:domain:ac701} на странице~\pageref{fig:domain:ac701:hdmi}. Контроллер конфигурируется
по протоколу I2C, путём записи управляющих регистров. Например, если цветовая субдискретизация была
проведена на предыдущем этапе в блоке AXI HDMI TX, то её можно отключить за отсутствием необходимости.

\subsection{Блок контроллера I2C}
\label{sec:structural:i2c}
Контроллер I2C используется для конфигурирования модулей камеры и HDMI. Данное решение продиктовано
аппаратными особенностями модулей камер и HDMI. Так как для их конфигурирования нужна лишь запись определенных регистров,
то использование более сложного и скоростного протокола, например SPI, было бы нецелесообразным.
