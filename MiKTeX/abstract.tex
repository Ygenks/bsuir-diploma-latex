\sectioncentered*{Реферат}
\thispagestyle{empty}

Дипломный проект представлен следующим образом. Электронные носители: 2 компакт-диска.
Чертёжный материал: 6 листов формата А1. Пояснительная записка: 87 страниц, \totfig{}~рисунка,
\tottab{}~таблиц, \totref{}~литературных источников, 3 приложений.

Ключевые слова: обработка видео, поточная обработка, FPGA, системы обработки видео,
параллелизм, высокое разрешение, модульность, аппаратный.

Объектом исследования и разработки является реализация аппаратной системы обработки видеоптока.

Целью дипломного проекта является разработка аппаратной системы, способной обрабатывать видео высокого
разрешения, с поддержкой модульности.

При разработке системы использовалась отладочная плата компании \en{Xilinx}, САПР \en{Vivado},
IP-ядра компании \en{Xilinx}. В основе спроектированной системы лежит модульный подход.

В результате разработки системы получена аппаратная платформа для обработки видео, обладающая высоким
быстродействием.

Данный проект может использоваться в качестве платформы для построения систем обработки видео,
что значительно сэкономит время их разработчикам.

Разработанная система является экономически эффективной, полностью окупающей вложенные средства.

Дипломный проект является завершённым, поставленная задача решена в полной мере, присутствует возможность
доработки проекта для применения в системах реального времени.

\clearpage
