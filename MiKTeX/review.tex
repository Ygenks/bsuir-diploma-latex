\thispagestyle{empty}

\begin{singlespace}

{\small
  \begin{center}
    \begin{minipage}{0.9\textwidth}
      \begin{center}
        {\normalsize РЕЦЕНЗИЯ}\\[0.2cm]
        на дипломный проект студента факультета компьютерных систем и сетей
        Учреждения образования <<Белорусский государственный университет информатики и радиоэлектроники>>\\
        Солдатенко Евгения Владимировича на тему: \\
        <<Аппаратная система обработки видеопотока>>
      \end{center}
    \end{minipage}\\
  \end{center}

Студент Солдатенко Е.В выполнил дипломный проект на шести листах графического материала
и на 87 страницах расчётно-пояснительной записки.

Тема проекта является актуальной и посвящена разработке системы аппаратной обработки видео.
Разработка данной системы обусловлена отсутствием модульной системы обработки видео,
поддерживающей обработку видео высокого разрешения.

Дипломный проект полностью соотвествует заданию.

Пояснительная записка построена логично, последовательно отражая все этапы разработки
системы в соответвующих разделах.

В пояснительной записке достаточно полно сделан сравнительный обзор существующих технологий
обработки видео, служащий основанием для дальнейшего обзора используемых в проекте средств.
Так же детально изучены принципы работы выбранной платформы, что говорит о заинтересованности
студента в понимании всех уровней абстракции системы.

Выбор некоторых компонентов системы основывается на расчётах, приведённых в пояснительной записке.
Приведённые в пояснительной записке расчёты произведены корректно.

По результатам дипломной работы спроектирована аппаратная платформа обработки видео и управляющая
программная часть. Выбор конечной реализации аппаратных блоков произведён аргументированно, в соответствии
с заданными требованиями. Архитектура конечной системы говорит о подготовленности студента к решению
сложных схемотехнических и программных проблем.

Разработанный проект является практически значимым, так как изначально предназначен для последующего
использования в системах обработки видео, приспособленных под конкретные задачи. Выбранная аппаратная
платформа позволяет использовать систему независимо от конкретной модели микросхемы, ограничиваясь
рамками семейства.

Существует несколько замечаний:
\begin{itemize}
  \item отсутствует описание настройки системного окружения для запуска ПО;
  \item недостаточно подробно описаны аналоги проектируемой системы.
\end{itemize}

Пояснительная записка и графический материал оформлены в соответствии с требованиями. Стиль
изложения материала отвечает академическим стандартам.

Дипломный проект выполнен грамотно, в полном соответствии с техническим заданием и заслуживает оценки десять баллов,
дипломник Солдатенко~Е.\,В. "--- присвоения квалификации инженера-системотехника.

  \vfill
  \noindent
  \begin{minipage}{0.4\textwidth}
    \begin{flushleft}
      Рецензент:\\
      ассистент кафедры ВМиП \\
      13.06.2018
    \end{flushleft}
  \end{minipage}
  \begin{minipage}{0.58\textwidth}
    \begin{flushright}
    \underline{\hspace*{2cm}} Т.\,А.~Рак \\
    \end{flushright}
  \end{minipage}
}

\end{singlespace}
\clearpage