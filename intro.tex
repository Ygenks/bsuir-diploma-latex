\sectioncentered*{Введение}
\addcontentsline{toc}{section}{ВВЕДЕНИЕ}
\label{sec:intro}

Решение многих задач в науке и инженерии требует обработки различного рода данных, одним из которых являются изображения.
Задачами обработки изображений занимается целая наука --- цифровая обработка сигналов (ЦОС),
изображения в которой рассматриваются как особый класс многомерных сигналов.

Высокая стоимость вычислительной техники в 60-70-е годы ограничивала применение
цифровых методов обработки сферами национальной безопасности, медициной, исследованием космоса.
Появление персональных компьютеров дало толчок в развитии ЦОС --- конечные пользователи
получили возможность решать задачи связанные с обработкой сигналов прямо на домашнем ПК,
при помощи различных подходов.

Обработка изображений в контексте ЦОС несет в себе ряд особенностей.
Во-первых, изображения характеризуются изменением пространственных параметров, тогда как звуковые сигналы описываются изменением временных параметров.
Во-вторых, объёмы информации, содержащейся в изображениях, очень велики.
Например, видеопоток несжатого RGB видео
с разрешением 1280х720 пикселей и частотой 50 кадров в секунду
занимает порядка 1280 $\cdot$ 720 $\cdot$ 3 $\cdot$ 50 $=$ 138,24 МБ памяти, когда как одна
секунда двухканального аудиосигнала с частотой дискретизации 44100 Гц
занимает 44100 $\cdot$ 2 $=$ 0,0882 МБ, что в 1567 раз больше. В-третьих, при оценке качества
изображения субъективная оценка доминирует над объективными критериями.

Несмотря на активное развитие рынка систем обработки видео, достаточно сложно найти
универсальную и модульную систему позволяющую обрабатывать видео высокого разрешения.
Существующие аппаратные реализации, как правило, <<заточены>> под решение одной конкретной задачи,
что делает практически невозможным переиспользование системы в случае смены условий решаемой проблемы.
Системы основанные на программной обработке до сих пор не могут обеспечить достаточной скорости,
и подходят исключительно для видео с низким фреймрейтом или разрешением. % Ссылка

% Ещё один абзац

Целью данного дипломного проекта является разработка аппаратной системы обработки
видеопотока, основными особенностями которой являются модульность и поддержка
обработки видео высокого разрешения. Модульность даёт возможность стороннему
разработчику дополнять поток обработки видео необходимыми блоками.
Поддержка видео высокого разрешения становится всё более актуальной:
большинство современных потребительских медиа устройств гарантированно
поддерживают съёмку видео в разрешении 1920x1080 и выше. % Надо дать какую то ссылку


В соответствии с целью были поставлены следующие задачи:
\begin{itemize}
  \item обзор подходов к построению систем обработки видео;
  \item обзор конкурентных решений в рамках выбранного подхода;
  \item выбор необходимых блоков для проектирования системы;
  \item разработка структурной схемы;
  \item реализация проекта в САПР;
  \item разработка конфигурационного кода для управления системой.
\end{itemize}

